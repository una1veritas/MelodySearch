\documentclass[11pt]{jreport}
\usepackage{latexsym}
\usepackage{mathrsfs}
\usepackage{amssymb}
%\usepackage{url}
%\usepackage{lscape}
\usepackage{graphics}
\usepackage{theorem}

%% for apple LaserWriter Series %%
%% 
\setlength{\topmargin}{-0.5in}
\setlength{\textwidth}{5.6in}
\setlength{\textheight}{8.8in}
\setlength{\oddsidemargin}{0.35in}
\setlength{\evensidemargin}{0in}

\usepackage{theorem}
\renewcommand{\baselinestretch}{1.25}
\setlength{\parskip}{0.25ex}
\renewcommand{\arraystretch}{0.85}
\begin{document}

有限アルファベットを $\Sigma$,その Kleene 閉包を $\Sigma^*$ で表す.
文字列 $s\in \Sigma^*$ の長さを $|s|$ で表す.
また長さ $0$ の空文字列を $\varepsilon$ で表す.

\subsection*{非決定性オートマトン}

\begin{defn}[非決定性有限オートマトン nondeterministic finite automata, NFA]
非決定性有限オートマトン(NFA) $M = (\Sigma, Q, \delta, q_0, F)$ とは,....

\vspace{2cm}

文字列 $s \in \Sigma^*$ を受け取った NFA $M$ が状態を $(q_0, \ldots, q_n)$ ただし $n = |S|$ と遷移するとき,
この列を $s$ に対する $M$ の計算といい,特に最後の状態 $q_n$ が $q_n \in F$ である計算を受理計算とよぶ.
一般に,非決定性有限オートマトンは一つの文字列に対して複数の計算を持つ.
ある $s\in\Sigma^*$ に対して $M$ に受理計算が存在するとき,$M$ は $s$ を受理するといい,
$M$ が受理する文字列すべての集合 $L(M) = \{ s \in \Sigma^* \mid \mbox{$M$ は $s$ を受理する}\}$ を $M$ が受理する言語という.
\end{defn}

\subsection*{正規表現}

\subsection*{NFA の計算}

ある文字列 $s\in\Sigma^*$ を NFA $M = (\Sigma, Q, \delta, q, F)$ が受理するかどうかを決定的なアルゴリズムで求めるためには,以下のように行う.

まず,遷移関係 $\delta \subseteq Q \times \Sigma \times Q$ を状態と文字から状態への集合 
\[
\delta(q, a) = \{q' \in Q \mid \mbox{ $(q, a, q') \in \delta$ }\}
\]
に拡張し,さらに状態の集合と文字から状態の集合への写像 $\tilde\delta: 2^Q \times \Sigma \to 2^Q$ に拡張する:
\[
\tilde\delta(S, a) = \bigcup_{q \in S} \delta(q, a)
\]
すると,有限オートマトン $\Sigma, 2^Q, \tilde\delta, \{q_0\}, F')$ は決定性有限オートマトンである.
ただし $F' = \{ S \subseteq 2^Q \mid S\cap F \neq \emptyset\}$. 


ドントケア * 記号(可変長ドントケア variable-length don't-care)と
複数文字集合(OR 記号,$\{\mathtt{a}, \mathtt{b}\}$ または $\mathtt{(a|b)}$ 等と書く)を含む文字列を
検索パターンとする NFA の計算は,最初に受理状態に達した時点で受理計算としてよい.
したがって,$\tilde\delta$ をもちいた
 * と複数文字集合からなるパターン照合を行う NFA の計算アルゴリズムは,以下のようになる.
 \begin{itemize}
 \item $S \leftarrow \{q_0\}$.
 \item ...
 \end{itemize}


\end{document}
